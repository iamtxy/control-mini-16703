\textbf{Part A}

The scaling frequency evaluates to
\[
\omega_0 = \sqrt{\frac{1(10/9 + 10)}{(10/9)\cdot 10}} = 1 .
\]

Then, the kinematic relations, 
\begin{align}
\dot{x}_1 &= x_3, \\
\dot{x}_2 &= x_4 .
\end{align}

Then, the remaining dynamics are
\begin{align}
\dot{x}_3 &= \frac{1}{J_1}
\left[-c(x_3 - x_4) - k(x_1 - x_2) + k_f I \right], \\
\dot{x}_4 &= \frac{1}{J_2}
\left[-c(x_4 - x_3) - k(x_2 - x_1) \right].
\end{align}

Substituting numerical values:
\begin{align}
\dot{x}_3 &= -0.9(x_1 - x_2) - 0.09(x_3 - x_4) + 0.9 I, \\
\dot{x}_4 &= \phantom{-}0.1(x_1 - x_2) + 0.01(x_3 - x_4).
\end{align}

Then, 
\[
A =
\begin{bmatrix}
0 & 0 & 1 & 0 \\
0 & 0 & 0 & 1 \\
-0.9 & 0.9 & -0.09 & 0.09 \\
0.1 & -0.1 & 0.01 & -0.01
\end{bmatrix},
\qquad
B =
\begin{bmatrix}
0 \\ 0 \\ 0.9 \\ 0
\end{bmatrix}.
\]

The output is the angular displacement of the external load:
\[
y = \varphi_2 = x_2 .
\]

Thus,
\[
C = \begin{bmatrix} 0 & 1 & 0 & 0 \end{bmatrix},
\qquad
D = \begin{bmatrix} 0 \end{bmatrix}.
\]

\[
\begin{aligned}
\dot{x} &= A x + B I, \\
y &= C x
\end{aligned}
\]


\newpage